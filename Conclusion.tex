
\section{Conclusions}

In this paper, we presented a simple and novel node sequence-based localization technique based on linear programming, LPSBL. 
In LPSBL, node sequences are used to uniquely identify distinct regions in the localization space. 
The reference node sequence is computed by using TOA measurements of acoustic signals between the acoustic source and the reference nodes.
The inequality constraint is constructed by processing the nodes sequence, then turn the sequence-based localization into linear programming problem. 
Since our system runs on COTS smartphones and supports spontaneous setup, 
 it has potential to enable a wide range of distributed acoustic source localization systems. 
 Besides the basic design, advanced LPSBL is proposed for further enhancing system robustness.
 Our system is verified and evaluated through analysis, extensive simulation as well as the test-bed experimentation.
 The test results have shown that the proposed method can effectively implement aoustic source localization with ad-hoc smartphone array.
% Our next step is to study the distributed localization method for ad-hoc smartphone array.
% Another future work is that further mining the information embedded in the node sequence to improve the robustness of localization system.

\section*{Acknowledgments}
This work is supported by the Fundamental Research Funds for the Central Universities (Grants No. DUT15QY05, DUT15QY51, No. DUT14ZD218 and DUT14QY29) and is partly supported by Natural Science Foundation of China (Grant No. 61272524).  Naigao Jin is the corresponding author.

