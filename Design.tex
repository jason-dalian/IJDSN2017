
\section{DESIGN}

In this section, we present the proposed LPSBL in detail.
The key ideas of LPSBL is that converting sequence-based localization into the linear feasibility problem.
After the basic LPSBL method is proposed, we describe the simplified LPSBL, incremental LPSBL and LPSBL with flip tolerance in the next subsection.
Finally, the computational complexity analysis of LPSBL is given.

\subsection{Basic LPSBL}

In this section, we introduce the basic sequence-based localization technique based on linear programming method.

Considering a sensor network in the 2D space with $N$ nodes, all nodes is $\bm{X} = \{ \bm{nod{e_1}}, \cdots ,\bm{nod{e_i}}, \cdots ,\bm{nod{e_N}}\}$, 
where any node $\bm{nod{e_i}}$ has its location coordinates denoted as $[{x_i},{y_i}]$. 
As showed in Fig. 2, an acoustic event occurs at $\bm{X_s}{\rm{ = [}}{x_s};{y_s}]$, ${d_i}$ is the distance from $nod{e_i}$ to the acoustic source $\bm{X_s}$.
The node sequence is determined by the distances from nodes to the acoustic source $\bm{X_s}$.
Given the following node sequence $NodeSeq( \cdots ,i,j, \cdots )$, ${d_i}<{d_j}$ can be inferred. We have the following inequality:
 \begin{equation} \label{equation_1}
(x_i-x_s)^2+(y_i-y_s)^2 < (x_j-x_s)^2+(y_j-y_s)^2
 \end{equation}
Then, we get
\begin{equation} \label{equation_2}
2(x_j-x_i)x_s+2(y_j-y_i)y_s<x_j^2-x_i^2+y_j^2-y_i^2
\end{equation}

Given the node sequences with $N$ node, we can get $N(N - 1)/2$ linear constraints. The
locations of nodes can be computed by solving the following linear feasibility problem:
\begin{equation}
\bm {{A}{X_s}}< \bm{{b}}
\end{equation}
where
\[\bm{A} = 
\left[
\begin{array}{lcr}
2x_2-2x_1 & 2y_2-2y_1 \\
2x_3-2x_2 & 2y_3-2y_2 \\
\quad  \quad      \vdots  &   \vdots  \\
2x_n-2x_{n-1} & 2y_n-2y_{n-1}
\end{array}
\right]
\]

\[\bm {b}=
\left[
\begin{array}{lcr}
x_2^2-x_1^2+y_2^2-y_1^2 \\
x_3^2-x_2^2+y_3^2-y_2^2 \\
  \quad \quad \quad  \vdots \\
x_n^2-x_{n-1}^2+y_n^2-y_{n-1}^2
\end{array}
\right]
\]


The problem of finding a feasible solution to a system of linear inequalities can be described as a linear programming problem in which the objective function is zero. 
We can find a solution to this feasibility program only if there is an embedding satisfying all of the constraints. 
We utilize the nodes sequences to set the inequality constraint.
The bound constraint in the linear programming problem is set based on the size of the network deployment area.
Describing the problem of sequence-based localization in the standard form of linear programming as
 \begin{equation} \label{6}
 \quad \quad \bm {\hat X_s} = \min {\bm{c^T}\bm{X_s}}
  \end{equation}
\begin{align*}
 s.t. \   \bm {{A}{X_s}} &\le \bm {b} 
\end{align*}
where $\bm {c}$ is zero vector.

To summarize, the basic LPSBL is presented in Algorithm 1. 
The input is the node sequences and locations of anchor; the output is the position of the acoustic source. 
% Step 1 sets the objective function of the optimization problem. Then, inequality constraints are constructed by the node sequence. Finally, LP problem is solved to achieve the position of the acoustic source.
\begin{algorithm}
\caption{LPSBL Method}
\KwIn { The position of $N$ nodes \\
\hspace{0.41in} The node sequence of the acoustic source for\\
\hspace{0.41in} $N$ reference nodes
}
\KwOut {Position of the acoustic source}

\textbf{Step 1:} Setting objective function: setting $\bf{c} \leftarrow \bf{0}$;

\textbf{Step 2:} Constructing inequality constraint: \\ 
\hspace{0.41in} setting $\bm{A}$ and $\bm{b}$ by processing the node sequence;\\
\For{$i \leftarrow 1$ \textbf{to} $N-1$}
{
\For{$j \leftarrow i+1$ \textbf{to} $N$}
{
According to inequality (\ref{equation_2}) to get the inequality constraint;
}
}

\textbf{Step 3:} Solve LP problem to get the target's position.
\end{algorithm}

 \vspace{-3mm}  
 \subsection{Simplified LPSBL}
 
% However, compared with the constraint of AD, the the constraint of AC is subject to flip.
   % Given the following node sequence $NodeSeq( \cdots ,i,j, \cdots )$, the uncertainty of ${d_i} < {d_j}$ can be described as : 
  % \begin{equation}
  % p({d_i} < {d_j})>\alpha
  % \end{equation}
  
  In this section, we consider the different effect of  $N(N - 1)/2$ constraints for the proposed LPSBL system.     
  We observes that the constraints of AC, CB and BD are tight in the $NodeSeq (A C B D)$.  
  Compared with non-adjacent nodes in the node sequence, the constraint constructed by the two adjacent nodes can contribute more to the localization accuracy. 
  Given the following node sequence $NodeSeq( \cdots ,i,j, \cdots )$ of $N$ nodes, 
  we present a simplified LPSBL, called LPSBL-Lite in Algorithm 2, that uses the adjacent nodes in the node sequence to construct $N-1$ constraints of the linear programming.
  The number of constraints is decreased from $N(N - 1)/2$  to $N-1$, that leads to the improvement of computational efficiency at the cost of localization performance.
\begin{algorithm}
\caption{Simplified LPSBL}
\KwIn { The position of $N$ nodes \\
\hspace{0.41in} The node sequence of the acoustic source for\\
\hspace{0.41in} $N$ reference nodes
}
\KwOut {Position of the acoustic source}

\textbf{Step 1:} Setting objective function: setting $\bf{c} \leftarrow \bf{0}$;

\textbf{Step 2:} Constructing inequality constraint: \\ 
\hspace{0.41in} setting $\bm{A}$ and $\bm{b}$ by processing the node sequence;\\
\For{$i \leftarrow 1$ \textbf{to} $N-1$}
{
\For{$j \leftarrow i+1$}
{
According to inequality (\ref{equation_2}) to set the inequality constraint;
}
}

\textbf{Step 3:} Solve LP problem to get the target's position.

\end{algorithm}

   \vspace{-3mm}  
\subsection{Incremental LPSBL }

In acoustic sensor network, after the nodes generate TOA mesaurements,
these TOA data reach the the sink node in a sequential way.
The basic idea of the proposed incremental LPSBL is to ultilize the available TOA data to construct constraints, then solve the linear program to achieve the position of the acoustic source.
As described in Algorithm 3, once the new TOA data come, incremental LPSBL add the one or two new constraints to update the position of the acoustic source.
\begin{algorithm} [b]
\caption{Incremental LPSBL Method}
 \KwIn {data package $pack_k$($ID_k$,$Location_k$,$TOA_k$) \\
 }
 \KwOut {Position of the acoustic source}

 \textbf{Step 1:} Initialization \\
                   \hspace{0.0in} (1) Sorting TOAs of the first two data package  \\
                   \hspace{0.0in} (2) Constructing the first constraint \\
\textbf{Step 2:}    \textbf{while} (the newest data package $pack_k$ reach)\\
                 %   \hspace{0.21in}\textbf{do:}\\
				%Constructing the inequality constraint according to the TOA of the newest data package:  \\
			    \hspace{0.0in} (1) Inserting the $TOA_k$ into the arrived TOA sequence \\
				\hspace{0.3in}by sorting algorithm\\
                   \hspace{0.0in} (2) Constructing one or two new constraints by $TOA_k$ \\ 
				  \hspace{0.28in} and its neighbors\\
                   \hspace{0.0in} (3) Solve the LP problem based on existing constraints\\
                   \textbf{end while}\\
 \end{algorithm}


 


\subsection{LPSBL with Flip Tolerance }

    \begin{figure}[!h]
    \centering
 \setlength{\abovecaptionskip}{-15pt}
                     %\vspace{1mm}  
 %\setlength{\belowcaptionskip}{-5pt}
    \includegraphics[height=4.7cm]{image/flip.eps} 
	\vspace{10mm}
    \caption{The Effect of Sequence Flip}
	\label{fig2}
    \vspace{-5mm}
    \end{figure}

For the sake of presentation, until now we have described LPSBL in an ideal case where a complete and perfect node sequence can be obtained. 
In this section, we describe how to make LPSBL work well under more realistic conditions. 

In the practical application, if two nodes are located too close to each other along the direction of event propagation, they detect the event almost simultaneously. 
In this case, the node ordering in the sequence may occur flip. 
For instance, the true sequence is $NodeSeq ( \cdots ,i,j, \cdots )$, but the detected sequence is $NodeSeq ( \cdots ,j,i, \cdots )$.
Algorithm 1 can find a solution to this feasibility program only if there is an embedding satisfying all of the constraints. 
However, it is impossible to find a feasible solution that satisfies all of the constraints when sequence flip occurs. 
For example, as showed in Fig. 2, the right middle region identifies by the node sequence $NodeSeq (D C A B)$. 
Once the order of node A and B occur flip in $NodeSeq (D C A B)$, the region corresponding to $NodeSeq (D C B A)$ does not exist, Algorithm 1 can not give the accurate estimation.
In this section, we propose the solution to address the problem of sequence flip using traditional convex relaxation techniques.


We thus introduce a slack variable ${\xi _{ij}}$ for each inequality constraint to allow for inequality violations.
Rewrite the inequality constraint (\ref{equation_2}) as
 \begin{equation} \label{10}
 d_i - d_j  \le  0 ; \  i,j \in S
 \end{equation}
 Relax the inequality constraint (\ref{10}) to
 \begin{equation} \label{11}
 d_i - d_j  \le  \xi _{ij} ; \  i,j \in S
 \end{equation}
 where
 \begin{math}
 \xi _{ij}  > 0
 \end{math}.
 
As a result, LPSBL with error tolerance can be formulated as a convex optimization problem with linear inequalities constraints as follows:
 \begin{equation} \label{5}
\quad \quad \quad \mathop {\min }\limits_{{x_i},{y_i}} \sum\limits_{(i,j) \in X} {{\xi _{ij}}};\  \xi _{ij} \ge 0\\
  \end{equation}
  \begin{align*}
 s.t.\   d_i - d_j  \le  \xi _{ij} \
\end{align*}
%where the objective function of optimization problem is the total amount of all slacks. 

In the optimization problem expressed by the equation \eqref{5}, the problem of inequality violations can be solved by introducing a slack variable for each inequality \cite{Boyd2004}.
Therefore, the proposed LPSBL can provide the robust estimation when the problem of sequence flip occurs.

 
\subsection{Complexity Analysis}
This section provides the complexity analysis for the proposed
LPSBL design. It needs to be emphasized that
LPSBL itself adopts an asymmetric design in which sensor
nodes need only to detect and report the events. Therefore,
we only analyze the computational cost on the node sequence
processing side, where resources are plentiful.

SBL: Calculating the Kendall’s Tau
between two sequences is $O(N^2)$ operations. Since the location sequence table is of size
$O(N^4)$, searching through it takes $O(N^6)$ operations~\cite{yedavalli2008sequence}.

LPSBL: The complexity of low dimensional linear programming
with L constraints is $O(L)$\cite{Griva2009}. The number of linear constraints is $N(N-1)/2$ in LPSBL,
where $N$ is the number of nodes. Thus, the overall computation
complexity of LPSBL can be written as $O(N^2)$.



