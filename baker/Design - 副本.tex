
\section{DESIGN}

In this section, we firstly introduce the basic SBL-LP method. After the basic method is proposed, then we describe the robust SBL-LP method in the next subsection \cite{sallai2011acoustic}.

\subsection{Basic LPSBL}

In this section, we introduce the Basic sequence-based localization technique based on linear programming method.

Considering a sensor network in the 2D space with $N$ nodes, all nodes is $X = \{ nod{e_1}, \cdots ,nod{e_i}, \cdots ,nod{e_N}\}$, 
where any node $nod{e_i}$ has its location coordinates denoted as $[{x_i},{y_i}]$. 
As showed in  in Figure 2, an acoustic event occurred at ${X_s}{\rm{ = [}}{x_s};{y_s}]$, ${d_i}$ is the distance from $nod{e_i}$ to the acoustic source $X_s$.
The node sequence is determined by the distances from nodes to the acoustic source $X_s$. 
Given the following node sequence $NodeSeq( \cdots ,i,j, \cdots )$, ${d_i} < {d_j}$ can be inferred. We have the following inequality:
 \begin{equation} \label{3}
(x_i-x_s)^2+(y_i-y_s)^2 < (x_j-x_s)^2+(y_j-y_s)^2
 \end{equation}
Then, we get
\begin{equation}
2(x_i-x_j)x_s+2(y_i-y_j)y_s<x_i^2-x_j^2+y_i^2-y_j^2
\end{equation}

Given the node sequences with N node, we can get N(N ? 1)/2 linear constraints. The
locations of nodes can be computed by solving the following
linear feasibility problem:
\begin{equation}
AX_s<b
\end{equation}
where
A =
\[
\left[
\begin{array}{lcr}
2x_2-2x_1 & 2y_2-2y_1 \\
2x_3-2x_2 & 2y_3-2y_2 \\
\quad       \vdots \\
2x_n-2x_{n-1} & 2y_n-2y_{n-1}
\end{array}
\right]
\]

b =
\[
\left[
\begin{array}{lcr}
x_2^2-x_1^2+y_2^2-y_1^2 \\
x_3^2-x_2^2+y_3^2-y_2^2 \\
\vdots \\
x_n^2-x_{n-1}^2+y_n^2-y_{n-1}^2
\end{array}
\right]
\]

The problem of finding a feasible solution
to a system of linear inequalities is a linear programming
problem in which the objective function is the zero. We
can find a solution to this feasibility program only if there
is an embedding satisfying all of the constraints. 
We utilize the nodes sequences to set the inequality constraint. The bound
constraint in the linear programming problem is set based
on the size of the network deployment area.

Describing the problem of sensor node localization in the standard form of linear programming as

 \begin{equation} \label{6}
 \quad \quad \hat X = \min {\bf{f}^T}X
  \end{equation}
\begin{align*}
 s.t. \   {A}X &\le {b} \\
       0 \le X & \le {G_{\max }}
\end{align*}
where $\bf{f}$ is zero vector.

To summarize, the LPSBL is presented in Algorithm
1. The input is the node sequences NodeSeqs and anchor locations; the output is the
locations of target nodes. Step 1 sets the objective function
of the optimization problem. Step 2 uses the node sequence
to get the inequality constraint. Each node sequence
is processed at step 3 to get the inequality constraint. Step 3
solves the LP problem to get the location of target nodes.
\begin{algorithm}
\caption{SBL-LP Method}
\textbf{Input:} The location of N Smartphones \\
\quad \quad \quad The location sequence of the active speaker for N reference nodes \\
\textbf{Output:} Location of the target\\

\textbf{Step 1:}  Setting objective function : setting

c = \bf {0}; \\

\textbf{Step 2:}\ Constructing inequality constraint: setting ${A_{EQU}}$ and ${b_{EQU}}$ according to the node sequence;\\
\For{$i \leftarrow 1$ \textbf{to} $N-1$}
{
\For{$j \leftarrow i+1$ \textbf{to} $N$}
{
According to inequality (4) to get inequality constraint;
}
}
\textbf{Step 3:}Solve LP problem to get the location of  target nodes; 
\end{algorithm}


\subsection{LPSBL with Error Tolerance }

For the sake of presentation, until now we have described LPSBL in an ideal case where a complete and perfect node sequence can be obtained. 
In this section, we describe how to make LPSBL work well under more realistic conditions firstly. 
Then, the computational complexity analysis of LPSBL is given.
In the practical application, if two nodes are located too close to each other along the direction of event propagation, they detect the scan almost simultaneously. 
In this case, the node ordering in the sequence may occur flip. 
For instance, the true sequence is $NodeSeq( \cdots ,i,j, \cdots )$, but the detected sequence is $NodeSeq( \cdots ,j,i, \cdots )$.
Algorithm 1 can find a solution to this feasibility program only if there is an embedding satisfying all of the constraints. 
However, it is impossible to find a feasible solution that satisfies all of the constraints when sequence flip occurs. 
In this section, we propose the solution to address the problem of sequence flip using convex relaxation techniques.


\begin{equation} \label{3-1}
-{\xi _{ij}}< SB-SA <{\xi _{ij}}; \
 \end{equation}


We thus introduce a slack variable ${\xi _{ij}}$ for each inequality constraint to allow for inequality violations.
 As a result, LPSBL with error tolerance can be formulated as a convex optimization problem with linear inequalities constraints as follows:
 \begin{equation} \label{5}
\quad \quad \quad \mathop {\min }\limits_{{x_i},{y_i}} \sum\limits_{(i,j) \in X} {{\xi _{ij}}};\  \xi _{ij} \ge 0\\
  \end{equation}
  \begin{align*}
 s.t.\  \left\| {SB - SA } \right\| \le \xi _{ij} \
\end{align*}
where the objective function of optimization problem is the total amount of all slacks.

In the optimization problem expressed by the equation \eqref{5}, because a slack variable is introduced for each inequality constraint to allow for inequality violations, the problem of sequence flip can be solved easily.
 Rewrite the inequality constraint (4) as
 \begin{equation} \label{10}
SB - SA  \ge  0 ; \  i,j \in S
 \end{equation}
 Relax the inequality constraint (10) to
  \begin{equation} \label{11}
SB - SA  \ge  -\varepsilon ; \  i,j \in S
 \end{equation}
 where
\begin{math}
\varepsilon  > 0
\end{math}.

Solution 1:
Relax the inequality constraint from \eqref{10} to \eqref{11}
 \begin{equation} \label{10}
{SB} - {SA} < \varepsilon
 \end{equation}

  \begin{equation} \label{11}
{d_j} - {d_i} < - \varepsilon
 \end{equation}


Solution 2:  Sequence flip may provide more constraints to event-driven localization. When the two nodes are close to each other along the scan direction, it means that the straight-line slope crossing the two nodes is close to the :
  \begin{equation} \label{12}
\left\| {SB - SA } \right\| \le \rho
 \end{equation}
where $\rho$ is given threshold parameter. Inequality \eqref{12} is equivalent to the following two linear constraints

\begin{eqnarray} \label{13}
\begin{array}{l}
 ({y_i} - {y_j}) - {\rm{tan}}\theta ({x_i} - {x_j}) \le \rho  \\
 {\rm{tan}}\theta ({x_i} - {x_j}) - ({y_i} - {y_j}) \le \rho  \\
 \end{array}
\end{eqnarray}



% \subsection{Robust LPSBL}

% Considering the uncertainty of the measurement,
% Given the following node sequence $NodeSeq( \cdots ,i,j, \cdots )$, the uncertainty of ${d_i} < {d_j}$ can be described as : 
% \begin{equation}
% p({d_i} < {d_j})>\alpha
% \end{equation}


% Considering the uncertainty of the location of anchor node and the uncertainty of the measurement together, we have
% \begin{equation}
% p((x_i-x_s)^2+(y_i-y_s)^2 < (x_j-x_s)^2+(y_j-y_s)^2)>\alpha
% \end{equation}
% where ${x_i,y_i}$ and ${x_j,y_j}$ are modelled as independent random variables following the normal distribution with the known parameter in the inequality.





